\chapter{Wave splitting}
\label{app: wave splitting}


In \parencite{Goda1976}, a technique is presented to resolve the incident and reflected waves from the records of composite waves. It is applicable to both regular and irregular trains of waves. Two simultaneous wave records are taken at adjacent locations, and all the amplitudes of the Fourier components are analyzed by the FFT technique. This technique can be very useful in this thesis and is therefore summarized below.

\section{Principle of resolution technique}
Suppose we have a multi-wave-reflection system of regular waves in a wave flume. Waves generated by the wave paddle propagate forward and are reflected by a test structure. The reflected waves propagate back to the wave paddle and are reflected again. After which the re-reflected waves propagate again to the test structure. This repeats itself until the waves are fully attenuated. The wave system can be regarded as a superposition of a number of waves propagating in the positive and negative direction of the systems axis x. The wave train propagating in the positive direction is called the incident waves and that in the negative direction is called the reflected waves. Let the amplitude of superposed incident waves be $a_i$ and that of reflected waves be $a_r$. Then these waves are described to have the general form of 
\begin{equation}
    \eta_i = a_i \cos(kx- \omega t + \epsilon_i)
\end{equation}
\begin{equation}
    \eta_r = a_r \cos(kx- \omega t + \epsilon_r)
\end{equation}
where $\eta_i$ and $\eta_r$ are the surface elevations of incident and reflected waves, $k$ is the wave number of $\frac{2\pi}{L}$, $\omega$ is the wave frequency of $\frac{2\pi}{T}$ with T being the wave period and $\epsilon_i$ and $\epsilon_r$ are the phase angle of the incident and reflected waves. Further, we suppose that the surface elevations are recorded at two adjacent stations of $x_1$ and $x_2=x_1+\Delta l$. Research have shown that the theory works best if $\Delta l = n \cdot \frac{L}{4}$, with $n$ being an integer. The observed profiles of composite waves at locations $x_1$ and $x_2$ will be 
\begin{equation}
    \eta_1 = (\eta_i + \eta_r)_{x=x_1} = A_1 \cos(\omega t) + B_1 \sin(\omega t)
\end{equation}
\begin{equation}
    \eta_2 = (\eta_i + \eta_r)_{x=x_2} = A_2 \cos(\omega t) + B_2 \sin(\omega t)
\end{equation}
where,
\begin{equation}
    A_1 = a_i \cos(\phi_i) + a_r \cos(\phi_r)
    \label{eq: A1wavesplitting}
\end{equation}
\begin{equation}
    B_1 = a_i \sin(\phi_i) - a_r \sin(\phi_r)
\end{equation}
\begin{equation}
    A_2 = a_i \cos(k \Delta l + \phi_i) + a_r \cos(k\Delta l + \phi_r)
\end{equation}
\begin{equation}
    B_2 = a_i \sin(k \Delta l + \phi_i) - a_r \sin(k\Delta l + \phi_r)
    \label{eq: B2wavesplitting}
\end{equation}
\begin{equation}
    \phi_i = k x_1 + \epsilon_i
\end{equation}
\begin{equation}
    \phi_r = k x_1 + \epsilon_r
\end{equation}
Equations \ref{eq: A1wavesplitting} to \ref{eq: B2wavesplitting} can be resolved to the extimate of the incident and the reflected wave amplitudes
\begin{equation}
    a_i = \frac{1}{2 |\sin(k\Delta l)|}\sqrt{(A_2 - A_1 \cos(k \Delta l) - B_1 \sin(k\Delta l)^2 + (B_2 + A_1 \sin(k \Delta l) - B_1 \cos(k\Delta l)^2}
\end{equation}
\begin{equation}
    a_i = \frac{1}{2 |\sin(k\Delta l)|}\sqrt{(A_2 - A_1 \cos(k \Delta l) + B_1 \sin(k\Delta l)^2 + (B_2 - A_1 \sin(k \Delta l) - B_1 \cos(k\Delta l)^2}
\end{equation}


In the calculation, the dispersion relation of the following is presumed to hold:
\begin{equation}
    \omega ^2 = gk \tanh(kh)
\end{equation}
which is according to linear wave theory. Actual wave profiles usually contain some higher harmonics. When non-linearities in the waves are increased, this technique becomes less and less realistic.