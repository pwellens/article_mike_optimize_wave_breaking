\chapter{Python scripts for setting up simulations}
\label{app: scripts for setting up simulations}

% \lstset{language=python,
%                 basicstyle=\ttfamily,
%                 numbers=left,
%                 breaklines=true,
%                 stepnumber=1,
%                 showstringspaces=false,
%                 tabsize=1,
%                 breakatwhitespace=false,
%                 keywordstyle=\color{blue}\ttfamily,
%                 stringstyle=\color{red}\ttfamily,
%                 commentstyle=\color{applegreen}\ttfamily,
% }
% \begin{mdframed}[backgroundcolor=light-gray, roundcorner=10pt,leftmargin=1, rightmargin=1, innerleftmargin=0, innertopmargin=7,innerbottommargin=0, outerlinewidth=1, linecolor=light-gray]
% \begin{lstlisting}[linewidth=\columnwidth,caption=Functions used in setting up simulations ., label=script:setting up simulations]

    
% \end{lstlisting}
% \end{mdframed}

\lstset{language=python,
                basicstyle=\ttfamily,
                numbers=left,
                breaklines=true,
                stepnumber=1,
                showstringspaces=false,
                tabsize=1,
                breakatwhitespace=false,
                keywordstyle=\color{blue}\ttfamily,
                stringstyle=\color{red}\ttfamily,
                commentstyle=\color{applegreen}\ttfamily,
}
\begin{mdframed}[backgroundcolor=light-gray, roundcorner=10pt,leftmargin=1, rightmargin=1, innerleftmargin=0, innertopmargin=7,innerbottommargin=0, outerlinewidth=1, linecolor=light-gray]
\begin{lstlisting}[linewidth=\columnwidth,caption=Script used for creating directories on cluster ., label=script: creating directories script]

import numpy as np
import os

parameters = np.loadtxt('../input_files/parameters_DE.csv', skiprows=0, delimiter=',')
number_of_configurations = len(parameters)
print('making directory for {} configurations and a wave only case'.format(number_of_configurations))


if not os.path.exists('//storage04/marclus4/home/mblom/mike_simulaties/03_TESTS/optimization_clusterfolder/ComFLOW_simulations'):
    os.makedirs('//storage04/marclus4/home/mblom/mike_simulaties/03_TESTS/optimization_clusterfolder/ComFLOW_simulations/wave_only/input_files')


for i, geometry in enumerate(parameters):
    config = int(geometry[0])
    os.makedirs('//storage04/marclus4/home/mblom/mike_simulaties/03_TESTS/optimization_clusterfolder/ComFLOW_simulations/configuration_{}/input_files'.format(config))


    
\end{lstlisting}
\end{mdframed}

\lstset{language=python,
                basicstyle=\ttfamily,
                numbers=left,
                breaklines=true,
                stepnumber=1,
                showstringspaces=false,
                tabsize=1,
                breakatwhitespace=false,
                keywordstyle=\color{blue}\ttfamily,
                stringstyle=\color{red}\ttfamily,
                commentstyle=\color{applegreen}\ttfamily,
}
\begin{mdframed}[backgroundcolor=light-gray, roundcorner=10pt,leftmargin=1, rightmargin=1, innerleftmargin=0, innertopmargin=7,innerbottommargin=0, outerlinewidth=1, linecolor=light-gray]
\begin{lstlisting}[linewidth=\columnwidth,caption=Script to put job.sh file in all directories ., label=script: job file in directory]

import os
import shutil

def find_files(filename, search_path, replacement_file):
    result = []

    for root, dir, files in os.walk(search_path):

        #replace jobfile when old jobfile is already there
        if filename in files:
            print('job.sh replaced in {}'.format(root))
            os.remove(os.path.join(root,filename))
            shutil.copy(replacement_file,root)

        #place jobfile when jobfile is NOT ther
        if filename not in files and 'configuration' in root and 'input_files' not in root:
            print('job.sh put in {}'.format(root))
            shutil.copy(replacement_file, root)

        elif filename not in files and 'wave_only' in root and 'input_files' not in root:
            print('job.sh put in {}'.format(root))
            shutil.copy(replacement_file, root)

    return result


find_files('job.sh',  '//storage04/marclus4/home/mblom/mike_simulaties/03_TESTS/optimization_clusterfolder/ComFLOW_simulations','../input_files/job.sh')
    
\end{lstlisting}
\end{mdframed}

\lstset{language=python,
                basicstyle=\ttfamily,
                numbers=left,
                breaklines=true,
                stepnumber=1,
                showstringspaces=false,
                tabsize=1,
                breakatwhitespace=false,
                keywordstyle=\color{blue}\ttfamily,
                stringstyle=\color{red}\ttfamily,
                commentstyle=\color{applegreen}\ttfamily,
}
\begin{mdframed}[backgroundcolor=light-gray, roundcorner=10pt,leftmargin=1, rightmargin=1, innerleftmargin=0, innertopmargin=7,innerbottommargin=0, outerlinewidth=1, linecolor=light-gray]
\begin{lstlisting}[linewidth=\columnwidth,caption= Script for creating comflow.cfi and  breakwater.in files ., label=script: creating comflow.cfi and breakwater.in]

import numpy as np
import os
import time
import matplotlib.pyplot as plt
from running_simulations_functions import CoB, parabolic_beach, radius_of_pitch_inertia, bathymetry2geodef, restoring_pitch_moment

filedir = os.getcwd()

# ---- Geometries input:
parameters = np.loadtxt(filedir + '/../input_files/parameters_DE.csv', skiprows=0, delimiter=',')
rho_w = 1000 #[kg/m3] mass water
show_figure = 0

number_of_configurations = len(parameters)
for i, geometry in enumerate(parameters):
    configuration   = int(geometry[0])
    T               = geometry[1]
    W               = geometry[2]
    front_fraction  = geometry[3]
    top_fraction    = geometry[4]
    radius          = geometry[5]
    WL              = geometry[6]

    ### PART 1: CALCULATE PARABOLIC BEACH
    if top_fraction != 1 and front_fraction != 1:
        # define parabolic beach through 3 points
        h, x = parabolic_beach(radius, T, W, front_fraction, top_fraction, WL)
    else:
        h = [0, 0]
        x = h



    ### PART 2: MAKE INPUT FILE comflow.cfi
    CoBx, CoBz, mass_bw, area_underwater = CoB(T, W, front_fraction, top_fraction, WL, h, x)
    print(CoBx, CoBz)
    k_yy = radius_of_pitch_inertia(T,W,front_fraction,top_fraction,WL,rho_w, h,x, CoBx, CoBz)
    c_dd = restoring_pitch_moment(W, top_fraction, WL, h ,x, CoBx, CoBz, area_underwater)

    input_cfi_file          = open(filedir + '/../input_files/comflow.cfi', 'r')
    input_cfi_file_data     = input_cfi_file.read()

    input_cfi_file_data = input_cfi_file_data.replace('[!TAG:wave_onlytag1]', '')
    input_cfi_file_data = input_cfi_file_data.replace('[!TAG:wave_onlytag2]', '')

    input_cfi_file_data = input_cfi_file_data.replace('[!TAG:coupling_switch1]', '<coupling>')
    input_cfi_file_data = input_cfi_file_data.replace('[!TAG:coupling_switch2]', '</coupling>')
    input_cfi_file_data = input_cfi_file_data.replace('[!TAG:subgrid_switch1]', '<subgrid')
    input_cfi_file_data = input_cfi_file_data.replace('[!TAG:subgrid_switch2]', '</subgrid>')


    input_cfi_file_data = input_cfi_file_data.replace('[!TAG:bot1]', str(-T-WL-20))
    input_cfi_file_data = input_cfi_file_data.replace('[!TAG:top1]', str(-WL+20))
    input_cfi_file_data = input_cfi_file_data.replace('[!TAG:W1]', str(405 - W - 20))

    input_cfi_file_data = input_cfi_file_data.replace('[!TAG:bot2]', str(-T-WL-10))
    input_cfi_file_data = input_cfi_file_data.replace('[!TAG:top2]', str(-WL+10))
    input_cfi_file_data = input_cfi_file_data.replace('[!TAG:W2]', str(405 - W - 10))

    input_cfi_file_data = input_cfi_file_data.replace('[!TAG:bot3]', str(-T-WL-5))
    input_cfi_file_data = input_cfi_file_data.replace('[!TAG:top3]', str(-WL+5))
    input_cfi_file_data = input_cfi_file_data.replace('[!TAG:W3]', str(405 - W - 5))
    if T>5:
        input_cfi_file_data = input_cfi_file_data.replace('[!TAG:subgrid_switch3]', '<!--subgrid')
        input_cfi_file_data = input_cfi_file_data.replace('[!TAG:subgrid_switch4]', '</subgrid-->')
    else:
        input_cfi_file_data = input_cfi_file_data.replace('[!TAG:subgrid_switch3]', '<subgrid')
        input_cfi_file_data = input_cfi_file_data.replace('[!TAG:subgrid_switch4]', '</subgrid>')

    input_cfi_file_data = input_cfi_file_data.replace('[!TAG:startx]', str(405 - W*top_fraction))
    input_cfi_file_data = input_cfi_file_data.replace('[!TAG:startz]', str(-WL))


    input_cfi_file_data = input_cfi_file_data.replace('[!TAG:mass]', str(mass_bw))
    input_cfi_file_data = input_cfi_file_data.replace('[!TAG:k_yy]', str(k_yy))
    input_cfi_file_data = input_cfi_file_data.replace('[!TAG:CoGx]', str(CoBx))
    input_cfi_file_data = input_cfi_file_data.replace('[!TAG:CoGz]', str(CoBz))
    input_cfi_file_data = input_cfi_file_data.replace('[!TAG:CoGx_to_xfender]', str(W*top_fraction - CoBx))

    # if WL>0:
    input_cfi_file_data = input_cfi_file_data.replace('[!TAG:CoGz_to_zfender]', str(CoBz - WL))
    # else:
    #     input_cfi_file_data = input_cfi_file_data.replace('[!TAG:CoGz_to_zfender]', str(-CoBz - abs(WL)))

    # if h[0] < WL: #if slope crosses waterline
    #input_cfi_file_data = input_cfi_file_data.replace('[!TAG:CoGz_to_zfender]', str(-CoBz - abs(WL)))

    # else:
    #     input_cfi_file_data = input_cfi_file_data.replace('[!TAG:CoGz_to_zfender]', str(CoBz + abs(WL)))


    fp4 = open('//storage04/marclus4/home/mblom/mike_simulaties/03_TESTS/optimization_clusterfolder/ComFLOW_simulations/configuration_{}/input_files/comflow.cfi'.format(configuration), 'w')
    fp4.write(input_cfi_file_data)
    fp4.close()
    print('comflow.cfi file made for configuration {}'.format(configuration))
    print('')

    ### PART 3: MAKE GEOMETRY FILE
    # plot breakwater
    plt.figure(figsize=([5,1.5]))
    ax = plt.subplot(1,1,1)
    ax.plot((x[0]-5, x[0]+W+5), (0,0), '-b')                            #waterline
    ax.plot(x,h-WL,'-k',lw=2)                                           #slope
    ax.plot((x[0]+W,x[0]+W),(-T-WL,0-WL),'-k',lw=2)                     #backside
    ax.plot((x[-1], x[-1]+top_fraction*W),(0-WL,0-WL),'-k',lw=2)        #top (c)
    ax.plot((x[0], x[0]),(h[0]-front_fraction*T-WL,h[0]-WL),'-k',lw=2)  #frontside (a)
    ax.plot((x[0],x[0]+W),(-T-WL,-T-WL),'-k',lw=2)                      #bottom
    ax.set_aspect(aspect='equal')
    ax.set_xlabel(('x [m]'))
    ax.set_ylabel(('z [m]'))
    plt.savefig('//storage04/marclus4/home/mblom/mike_simulaties/03_TESTS/optimization_clusterfolder/ComFLOW_simulations/configuration_{}/input_files/breakwater_geometry{}.pdf'.format(configuration,configuration), dpi=150)
    if show_figure == 1:
        plt.show()
    plt.close()

    bathymetry2geodef(x, h, "breakwater_geometry.in", configuration, T, W, front_fraction, top_fraction, WL)



### PART 4: INPUT FILE comflow.cfi FOR WAVE ONLY CASE
input_cfi_file          = open(filedir + '/../input_files/comflow.cfi', 'r')
input_cfi_file_data     = input_cfi_file.readlines()

# input_cfi_file_data = input_cfi_file_data.replace('[!TAG:subgrid_switch1]', '<!--subgrid')
# input_cfi_file_data = input_cfi_file_data.replace('[!TAG:subgrid_switch2]', '</subgrid-->')

for i, line in enumerate(input_cfi_file_data):
    if '[!TAG:subgrid_switch1]' in line:
        input_cfi_file_data[i] = input_cfi_file_data[i].replace('[!TAG:subgrid_switch1]', '<!--subgrid')
        input_cfi_file_data[i] = input_cfi_file_data[i].replace('[!TAG:subgrid_switch2]', '</subgrid-->')
    if '<coupling>' in line:
        index_begin_erase = i
    if '</coupling>' in line:
        index_end_erase = i

del input_cfi_file_data[index_begin_erase:index_end_erase+1]

fp4 = open('//storage04/marclus4/home/mblom/mike_simulaties/03_TESTS/optimization_clusterfolder/ComFLOW_simulations/wave_only/input_files/comflow.cfi'.format(i), 'w')
fp4.writelines(input_cfi_file_data)
fp4.close()
print('comflow.cfi file made for wave_only')



    
\end{lstlisting}
\end{mdframed}

\lstset{language=python,
                basicstyle=\ttfamily,
                numbers=left,
                breaklines=true,
                stepnumber=1,
                showstringspaces=false,
                tabsize=1,
                breakatwhitespace=false,
                keywordstyle=\color{blue}\ttfamily,
                stringstyle=\color{red}\ttfamily,
                commentstyle=\color{applegreen}\ttfamily,
}
\begin{mdframed}[backgroundcolor=light-gray, roundcorner=10pt,leftmargin=1, rightmargin=1, innerleftmargin=0, innertopmargin=7,innerbottommargin=0, outerlinewidth=1, linecolor=light-gray]
\begin{lstlisting}[linewidth=\columnwidth,caption=Functions used in setting up simulations ., label=script:setting up simulations]
import numpy as np
import sys
import matplotlib.pyplot as plt
from sympy import symbols, Eq, solve
import os


def parabolic_beach(radius, T, W, front_fraction, top_fraction, WL):
    T_acc = front_fraction*T
    W_acc = top_fraction*W

    zA = T_acc-T
    xA = -(W-W_acc)

    zB = 0
    xB = 0

    x = [xA, xB]
    z = [zA, zB]

    h, k = symbols('h k')

    eq1 = Eq( ((xA - h)**2 + (zA - k)**2), radius**2 )
    eq2 = Eq( ((xB - h)**2 + (zB - k)**2), radius**2 )

    sol_dict = solve((eq1, eq2), (h, k))

    for i in range(2):
        if sol_dict[0][-1]>sol_dict[1][-1]:
            uppercenter =  sol_dict[0]
            downcenter = sol_dict[1]
        else:
            uppercenter = sol_dict[1]
            downcenter = sol_dict[0]


    x_line = np.linspace(xA, xB, 1000)
    z_line = []
    for x_single in x_line:
        z_line.append(downcenter[1] + 0.5*np.sqrt(float(4*downcenter[1]**2 - 4*(downcenter[1]**2 - radius**2 + (x_single-downcenter[0])**2))))



    return z_line, x_line

def CoB(T,W,front_fraction, top_fraction, WL, h ,x):

    if abs(WL)< T*(1-front_fraction) or WL>0: #if waterline is above or on slope
        if WL>0:
            WL_i = 0
        else:
            WL_i = WL
        x_i = x + W * (1 - top_fraction)
        h_i = h + T * (1 - front_fraction)

        dx = x_i[1] - x_i[0]

        #area 1: area unders slope beneath the waterline
        #area 2: area under slope above the waterline
        area_1 =0
        area_2=0
        area_z_sum = 0
        area_x_sum = 0
        area_x2_sum = 0
        for i, single_h in enumerate(h_i):
            if single_h<h_i[-1]-abs(WL_i):
                A = h_i[i]*dx     #squared meters of cell
                d = 0.5*h_i[i]    #distance to middle cell

                area_1 += A
                area_z_sum += A*d
                area_x_sum += A*x_i[i]

            elif single_h>h_i[-1]-abs(WL_i):
                A = dx* (T*(1-front_fraction) - abs(WL_i))
                d = 0.5* (T*(1-front_fraction) - abs(WL_i))

                area_2 += A
                area_x2_sum += A*x_i[i]

        z1 = 0
        x1 = 0
        if area_1 != 0:
            z1 = area_z_sum/area_1
            x1 = area_x_sum/area_1


        z2 = 0.5* (T*(1-front_fraction) - abs(WL_i))
        x2 = 0
        if area_2 != 0:
            x2 = area_x2_sum/area_2


        #area_3 (right next to slope, under horizontal top part)
        area_3 = (T*(1-front_fraction)-abs(WL_i))*W*top_fraction
        z3 = z2
        x3 = W - 0.5*W*top_fraction

        #area 4 (underneath box)
        area_4 = W*T*front_fraction
        z4 = -0.5*T*front_fraction
        x4 = 0.5*W

        CoBx = (area_1*x1 + area_2*x2 + area_3*x3 + area_4*x4)/(area_1 + area_2 + area_3 + area_4)
        CoBz = (area_1*z1 + area_2*z2 + area_3*z3 + area_4*z4)/(area_1 + area_2 + area_3 + area_4)
        mass_bw = (area_1 + area_2 + area_3 + area_4) * 1000

        area_underwater = area_1 + area_2 + area_3 + area_4

        CoBz += T * front_fraction



    else: #if waterline is under slope
        area_4 = (T-abs(WL))*W
        x4 = 0.5*W
        z4 = 0.5*(T-abs(WL))

        CoBx = x4
        CoBz = z4
        mass_bw = (area_4) * 1000

        area_underwater = area_4

    #change to domain space
    CoBx+=-W * (1 - top_fraction)
    CoBz -= T



    return CoBx, CoBz, mass_bw, area_underwater

def restoring_pitch_moment(W, top_fraction, WL, h ,x, CoBx, CoBz, area_underwater):

    if h[0]<WL:
        #find index where slope is crossing waterline
        index_wl = np.absolute(h - WL).argmin()

        L1 = abs(x[index_wl])
        L2 = W*top_fraction
        L = L1 + L2
        d = L1 + 0.5*L2 - CoBx

        #use parallel axes theorem to find total area moment of inertia (per unit width)
        I_yy = L**3/12 + L*d**2

    else:
        L = W
        d = (W*(1-top_fraction) + CoBx) - W/2 #distance from beginning geometry to CoBx - distance from beginning geometry to halve width

        I_yy = L**3/12 + L*d**2
    print('I_yy = {} m4'.format(round(I_yy,2)))
    BM = I_yy/area_underwater
    print('BM = {} m'.format(round(BM,2)))

    KB = CoBz
    KG = KB
    GM = KB + BM - KG
    c_dd = 1000*9.81*area_underwater*GM

    print('restoring pitch moment = {} Nm/unit width'.format(round(c_dd,2)))
    return c_dd

def radius_of_pitch_inertia(T,W,front_fraction,top_fraction,WL,rho_w, h,x, CoBx, CoBz):
    # area 3
    area_3 = 0
    if front_fraction!=1 and top_fraction!=1: #otherwise slope doesnt exist
        #area 1
        area_1 = W * top_fraction * T * (1 - front_fraction)
        d_x = 0.5*W*top_fraction - CoBx
        d_z = T*(0.5-0.5*front_fraction) - CoBz
        d = np.sqrt( float((d_x)**2 + (d_z)**2) )
        I_yy_1 = ((1/12)* ((W*top_fraction)**2 + (T*(1-front_fraction))**2) + d**2)


        #area 2
        area_2 = W * T * front_fraction
        d_x = -(0.5*W-W*top_fraction) - CoBx
        d_z = 0.5*T*front_fraction - CoBz
        d = np.sqrt(float((d_x) ** 2 + (d_z) ** 2))
        I_yy_2 = ((1/12)* (W**2 + (T*front_fraction)**2) + d**2)

        #area 3
        #define 0 at beginning of slope
        h += T * (1 - front_fraction)

        dx = x[1]-x[0]
        I_yy_3 = 0
        area_3 = 0
        for i, single_h in enumerate(h):
            d_x = abs(x[i] - CoBx)
            d_z = (T*front_fraction + 0.5*h[i]) - CoBz
            A = dx * h[i]
            d = np.sqrt( float((d_x) ** 2 + (d_z) ** 2) )
            I_yy_3 += ((1 / 12) * (abs(single_h) ** 2 + dx ** 2)*A + d**2*A)
            area_3 += abs(single_h * dx)
        I_yy_3=I_yy_3/area_3

        k_yy = np.sqrt( float( (I_yy_1*area_1 + I_yy_2*area_2 + I_yy_3*area_3) / (area_1+area_2+area_3)) )

    else:
        k_yy = np.sqrt((1/12)*(W**2 + T**2))

    return k_yy


def bathymetry2geodef(x,z,filename, config, T,W,front_fraction,top_fraction,WL):
    # computational domain
    zmin = np.amin(z)
    zmax = np.amax(z)
    ymin = -1.
    ymax =  1.
    xmin = x[0]
    xmax = x[-1]

    extents = [xmin, xmax, ymin, ymax, zmin, zmax]

    #print("- extents: {}".format(extents))

    filedir = os.getcwd()

    with open("//storage04/marclus4/home/mblom/mike_simulaties/03_TESTS/optimization_clusterfolder/ComFLOW_simulations/configuration_{}/input_files/breakwater_geometry.in".format(config),"w") as file:
        if top_fraction != 1 and front_fraction != 1:
            for i in range(1,x.size):
                if z[i]<zmin+1e-8:
                    file.write("2 0 0\n")
                    file.write(" ".join(map(str, [ x[i-1], ymin, zmin ]))+"\n")
                    file.write(" ".join(map(str, [ x[i-1], ymin, z[i-1] ]))+"\n")
                    file.write(" ".join(map(str, [ x[i  ], ymin, zmin ]))+"\n")
                    file.write(" ".join(map(str, [ x[i-1], ymax, zmin ]))+"\n")
                    file.write(" ".join(map(str, [ x[i-1], ymax, z[i-1] ]))+"\n")
                    file.write(" ".join(map(str, [ x[i  ], ymax, zmin ]))+"\n")
                elif z[i-1]<zmin+1e-8:
                    file.write("2 0 0\n")
                    file.write(" ".join(map(str, [ x[i  ], ymin, zmin ]))+"\n")
                    file.write(" ".join(map(str, [ x[i-1], ymin, zmin ]))+"\n")
                    file.write(" ".join(map(str, [ x[i  ], ymin, z[i] ]))+"\n")
                    file.write(" ".join(map(str, [ x[i  ], ymax, zmin ]))+"\n")
                    file.write(" ".join(map(str, [ x[i-1], ymax, zmin ]))+"\n")
                    file.write(" ".join(map(str, [ x[i  ], ymax, z[i] ]))+"\n")
                else:
                    file.write("3 0 0\n")
                    file.write(" ".join(map(str, [ x[i-1], ymin, zmin]))+"\n")
                    file.write(" ".join(map(str, [ x[i  ], ymin, zmin]))+"\n")
                    file.write(" ".join(map(str, [ x[i  ], ymax, zmin]))+"\n")
                    file.write(" ".join(map(str, [ x[i-1], ymax, zmin]))+"\n")
                    file.write(" ".join(map(str, [ x[i-1], ymin, z[i-1] ]))+"\n")
                    file.write(" ".join(map(str, [ x[i  ], ymin, z[i  ] ]))+"\n")
                    file.write(" ".join(map(str, [ x[i  ], ymax, z[i  ] ]))+"\n")
                    file.write(" ".join(map(str, [ x[i-1], ymax, z[i-1] ]))+"\n")
        #BOTTOM BLOCK
        if front_fraction>0:
            file.write("3 0 0\n")
            file.write(" ".join(map(str, [x[0],                     ymin,   z[0]-T*front_fraction])) + "\n")
            file.write(" ".join(map(str, [x[0]+W,                   ymin,   z[0]-T*front_fraction])) + "\n")
            file.write(" ".join(map(str, [x[0]+W,                   ymax,   z[0]-T*front_fraction])) + "\n")
            file.write(" ".join(map(str, [x[0],                     ymax,   z[0]-T*front_fraction])) + "\n")
            file.write(" ".join(map(str, [x[0],                     ymin,   z[0]])) + "\n")
            file.write(" ".join(map(str, [x[0]+W,                   ymin,   z[0]])) + "\n")
            file.write(" ".join(map(str, [x[0]+W,                   ymax,   z[0]])) + "\n")
            file.write(" ".join(map(str, [x[0],                     ymax,   z[0]])) + "\n")

        #AFT BLOCK
        if top_fraction>0:
            file.write("3 0 0\n")
            file.write(" ".join(map(str, [x[-1],                    ymin,   z[0]])) + "\n")
            file.write(" ".join(map(str, [x[-1]+W*top_fraction,    ymin,   z[0]])) + "\n")
            file.write(" ".join(map(str, [x[-1]+W*top_fraction,    ymax,   z[0]])) + "\n")
            file.write(" ".join(map(str, [x[-1],                    ymax,   z[0]])) + "\n")
            file.write(" ".join(map(str, [x[-1],                    ymin,   0])) + "\n")
            file.write(" ".join(map(str, [x[-1]+W*top_fraction,    ymin,   0])) + "\n")
            file.write(" ".join(map(str, [x[-1]+W*top_fraction,    ymax,   0])) + "\n")
            file.write(" ".join(map(str, [x[-1],                    ymax,   0])) + "\n")

        # #CONNECTION BLOCK
        # if top_fraction>0 and front_fraction>0:
        #     file.write("3 0 0\n")
        #     file.write(" ".join(map(str, [x[-1],                    ymin,   z[0]-T*front_fraction])) + "\n")
        #     file.write(" ".join(map(str, [x[-1]+W*top_fraction,    ymin,   z[0]-T*front_fraction])) + "\n")
        #     file.write(" ".join(map(str, [x[-1]+W*top_fraction,    ymax,   z[0]-T*front_fraction])) + "\n")
        #     file.write(" ".join(map(str, [x[-1],                    ymax,   z[0]-T*front_fraction])) + "\n")
        #     file.write(" ".join(map(str, [x[-1],                    ymin,   z[0]])) + "\n")
        #     file.write(" ".join(map(str, [x[-1]+W*top_fraction,    ymin,   z[0]])) + "\n")
        #     file.write(" ".join(map(str, [x[-1]+W*top_fraction,    ymax,   z[0]])) + "\n")
        #     file.write(" ".join(map(str, [x[-1],                    ymax,   z[0]])) + "\n")
    
\end{lstlisting}
\end{mdframed}