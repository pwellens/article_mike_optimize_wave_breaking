\chapter{Python scripts for post-processing}
\label{app: scripts for post-processing}


% \lstset{language=python,
%                 basicstyle=\ttfamily,
%                 numbers=left,
%                 breaklines=true,
%                 stepnumber=1,
%                 showstringspaces=false,
%                 tabsize=1,
%                 breakatwhitespace=false,
%                 keywordstyle=\color{blue}\ttfamily,
%                 stringstyle=\color{red}\ttfamily,
%                 commentstyle=\color{applegreen}\ttfamily,
% }
% \begin{mdframed}[backgroundcolor=light-gray, roundcorner=10pt,leftmargin=1, rightmargin=1, innerleftmargin=0, innertopmargin=7,innerbottommargin=0, outerlinewidth=1, linecolor=light-gray]
% \begin{lstlisting}[linewidth=\columnwidth,caption=Functions used in setting up simulations ., label=script:setting up simulations]

    
% \end{lstlisting}
% \end{mdframed}


\lstset{language=python,
                basicstyle=\ttfamily,
                numbers=left,
                breaklines=true,
                stepnumber=1,
                showstringspaces=false,
                tabsize=1,
                breakatwhitespace=false,
                keywordstyle=\color{blue}\ttfamily,
                stringstyle=\color{red}\ttfamily,
                commentstyle=\color{applegreen}\ttfamily,
}
\begin{mdframed}[backgroundcolor=light-gray, roundcorner=10pt,leftmargin=1, rightmargin=1, innerleftmargin=0, innertopmargin=7,innerbottommargin=0, outerlinewidth=1, linecolor=light-gray]
\begin{lstlisting}[linewidth=\columnwidth,caption=Script that does the hydrodynamical post-processing ., label=script:hydro post processing]

from post_processing_functions import breakwater_parameters_in_resultdict, stable_check, mean_wave_drift_force, transmitted_wave_height, CoB_postprocess
import pickle


## INPUT
#name of the run folders
runs = ['run_23'] #IF MORE, INCREASE INPUT IN CUTOFFSECONDS!
path_to_data = r'//storage04/marclus4/home/mblom/mike_simulaties/03_TESTS/optimization_clusterfolder/ComFLOW_simulations_finished'

## PUT BREAKWATER PARAMETERS IN DICTIONARY
resultDict = breakwater_parameters_in_resultdict(runs, path_to_data)

## STABLE CHECK AND INPUT COMFLOW IN DICT
resultDict = stable_check(resultDict,runs)

#CoB in resultDict
resultDict = CoB_postprocess(resultDict,runs)

## PUT MEAN WAVE DRIFT FORCE IN DICTIONARY
cutoffSeconds = [125]
show_figures = 0
show_all_forcefigures= 0
resultDict = mean_wave_drift_force(resultDict, runs, cutoffSeconds, show_figures, show_all_forcefigures)

## TRANSMITTED WAVE HEIGHT
show_figures=0
resultDict = transmitted_wave_height(resultDict, runs, cutoffSeconds, show_figures)


## SAVE resultDict TO FILE
for indexRun, run in enumerate(runs):
    with open('../output_files/data_{}'.format(run), 'wb') as f:
        pickle.dump(resultDict[indexRun], f, pickle.HIGHEST_PROTOCOL)




\end{lstlisting}
\end{mdframed}


\lstset{language=python,
                basicstyle=\ttfamily,
                numbers=left,
                breaklines=true,
                stepnumber=1,
                showstringspaces=false,
                tabsize=1,
                breakatwhitespace=false,
                keywordstyle=\color{blue}\ttfamily,
                stringstyle=\color{red}\ttfamily,
                commentstyle=\color{applegreen}\ttfamily,
}
\begin{mdframed}[backgroundcolor=light-gray, roundcorner=10pt,leftmargin=1, rightmargin=1, innerleftmargin=0, innertopmargin=7,innerbottommargin=0, outerlinewidth=1, linecolor=light-gray]
\begin{lstlisting}[linewidth=\columnwidth,caption=Costs post-processing ., label=script:Costs post-processing]

from post_processing_functions import extract_result_data, mean_wave_drift_force_floating_island_strip, geometry_costs_function
import pickle

# This script calculates everything regarding the costs. The breakwater price and the reduction in mooring costs

runs = ['run_21']

#MOORING COST REDUCTION
H_i = 3
Fd_without_breakwater = mean_wave_drift_force_floating_island_strip(H_i)


for indexRun, run in enumerate(runs):
    data = extract_result_data(run)

    for indexConfig, config in enumerate(data.configuration):
        if config.stable == True:

            Fd_island = (12304*9.80665E3)/9/45                                         #drift force where island is designed for D3.3 S@S per unit width
            total_mooring_cost_strip = (20E6)/9                         #euros
            mooring_costs_per_unit_width = total_mooring_cost_strip/45  #euros

            fraction_H = config.H_t/H_i

            cost_island_mooring_with_breakwater = fraction_H**2 * mooring_costs_per_unit_width      #mooring costs per unit with breakwater

            costs_breakwater_mooring  = mooring_costs_per_unit_width*  (config.mean_wave_drift_force/Fd_island)   #mooring costs per unit breakwater itself
            total_costs_with_breakwater = cost_island_mooring_with_breakwater+costs_breakwater_mooring #per unit width
            #print(total_costs_with_breakwater)


            config.costs_breakwater_mooring             = costs_breakwater_mooring
            config.costs_island_mooring_with_Ht         = cost_island_mooring_with_breakwater
            config.reduction_mooring_costs              = mooring_costs_per_unit_width - total_costs_with_breakwater


#BREAKWATER CONSTRUCTION COSTS
rho_w = 1000 #[kg/m3] density water
for indexRun, run in enumerate(runs):
    #data = extract_result_data(run)
    for indexConfig, config in enumerate(data.configuration):
        if config.stable == True:

            #print('configuration {} has a reduction in mooring cost of {}'.format(config.number, round(config.reduction_mooring_costs, 2)))

            geometry_costs = geometry_costs_function(config.T,config.W,config.front_fraction,config.top_fraction,config.verschuiving,config.WL,config.mass,config.h,config.x)
            #print('configuration {} has a construction cost of {} ME'.format(config.number,round(geometry_costs*10**(-6),2)))

            config.geometry_costs = geometry_costs
            print('configuration {} resulted in a cost reduction of {} euro'.format(config.number,round(config.reduction_mooring_costs - geometry_costs,2)))


    with open('../output_files/data_{}'.format(run), 'wb') as f:
        pickle.dump(data, f, pickle.HIGHEST_PROTOCOL)




    
\end{lstlisting}
\end{mdframed}


\lstset{language=python,
                basicstyle=\ttfamily,
                numbers=left,
                breaklines=true,
                stepnumber=1,
                showstringspaces=false,
                tabsize=1,
                breakatwhitespace=false,
                keywordstyle=\color{blue}\ttfamily,
                stringstyle=\color{red}\ttfamily,
                commentstyle=\color{applegreen}\ttfamily,
}
\begin{mdframed}[backgroundcolor=light-gray, roundcorner=10pt,leftmargin=1, rightmargin=1, innerleftmargin=0, innertopmargin=7,innerbottommargin=0, outerlinewidth=1, linecolor=light-gray]
\begin{lstlisting}[linewidth=\columnwidth,caption=Functions used in post-processing ., label=script: post processing functions]
import numpy as np
import glob
from packages.configuration_structure.Config import Configuration, Run
import re
import comflow
import matplotlib.pyplot as plt
import pymarin_timedom
import h5py
from scipy import interpolate
import pickle
import sys
sys.path.append('./packages')
from FUNCTIONS import define_L
sys.path.append('../running_simulations_scripts')
from running_simulations_functions import parabolic_beach
from scipy.optimize import fsolve
sys.path.insert(1,'M:/My Documents/pythonProject/Optimization/optimization_clusterfolder/running_simulations_scripts')
from running_simulations_functions import CoB, parabolic_beach

def breakwater_parameters_in_resultdict(runs, path_to_data):

    csv_switch = 0
    resultDict = []
    for indexRun, run in enumerate(runs):
        path = path_to_data + '/{}'.format(run)

        print('receiving information from {}'.format(path))
        r1 = Run(name = run, path = path)

        for indexConfig, configFolder in enumerate(glob.glob(path + '/*')): #find different configuration folders
            #read parameter csv file and put in data struct
            if 'parameters_DE.csv' in configFolder:
                csv_switch=1
                for singleConfig in np.loadtxt(configFolder, skiprows=0, delimiter=','):

                    runpath = path+'/configuration_{}'.format(int(singleConfig[0]))
                    config = Configuration(number=int(singleConfig[0]), T=singleConfig[1], W=singleConfig[2], front_fraction= singleConfig[3], top_fraction = singleConfig[4],
                                            verschuiving = singleConfig[5], WL = singleConfig[6], runpath=runpath)
                    r1.configuration.append(config)
            elif 'parameters_DE.csv' in configFolder:
                print('place parameters_DE.csv file in folder')
        resultDict.append(r1)


    if csv_switch == 0:
        print('no parameter.csv file in folder!')


    return resultDict


def CoB_postprocess(resultDict,runs):
    for indexRun, run in enumerate(runs):
        for config in resultDict[indexRun].configuration:
            h, x = parabolic_beach(config.verschuiving, config.T, config.W, config.front_fraction, config.top_fraction, config.WL)
            CoBx, CoBz, mass_bw, area_underwater = CoB(config.T, config.W, config.front_fraction, config.top_fraction, config.WL, h, x)

            config.CoBx = CoBx
            config.CoBz = CoBz
            config.mass = mass_bw
            config.h = h
            config.x = x

            print('CoB calculated for configuration {}'.format(config.number))
    return resultDict


def stable_check(resultDict, runs):
    ## STABLE CHECK AND INPUT COMFLOW IN DICT
    for indexRun, run in enumerate(runs):
        stable = 0
        unstable = 0
        for config in resultDict[indexRun].configuration:

            #check what the input tmax is
            file = open(r'{}/input_files/comflow.cfi'.format(config.runpath), 'r')
            lines = file.readlines()
            for line in lines:
                if 'tmax' in line:
                    for linepart in line.split(' '):
                        if 'tmax' in linepart:
                            config.tmax = float(re.findall(r'\d+.\d',linepart)[0])
                if 'model' in line and 'height' in line:
                    for linepart in line.split(' '):
                        if 'height' in linepart:
                            config.waveHeight = float(re.findall(r'\d+.|\d', linepart)[0])
                        if 'period' in linepart:
                            config.wavePeriod = float(re.findall(r'\d+.|\d', linepart)[0])
            try:
                file = open(r'{}/proc_1/tictoc.dat'.format(config.runpath), 'r')
            except:
                config.stable = False
                print('configuration {} has no tictoc.dat'.format(config.number))
                continue
            lines = file.readlines()
            tictoc_lastline = lines[-1].split()
            final_time_simulation = float(tictoc_lastline[1])


            #if final time is within two seconds of target time, simulation was stable
            if abs(final_time_simulation-config.tmax) < 2:
                config.stable = True
                stable+=1
            else:
                config.stable = False
                print('configuration {} is unstable, ran until t = {} sec'.format(config.number,round(final_time_simulation)))
                unstable+=1
        resultDict[indexRun].totalStable = stable
        resultDict[indexRun].totalUnstable = unstable

        print('{} stable simulations in {}'.format(stable, run))
        print('{} unstable simulations in {}'.format(unstable, run))
    return resultDict


def mean_wave_drift_force(resultDict, runs, cutoffSeconds, show_figures, show_all_forcefigures):

    for indexRun, run in enumerate(runs):
        for config in resultDict[indexRun].configuration:
            # IF STABLE, THEN CALCULATE MEAN WAVE DRIFT FORCE:
            if config.stable:
                forces = comflow.TabularData(config.runpath + '\proc_1\interactive_motion0001.dat')

                time = forces["Time"]

                start = cutoffSeconds[indexRun]
                start_idx = np.abs(time - start).argmin()
                #stop_idx = np.abs(time - end).argmin()
                # time = time[start_idx:stop_idx]
                # Fx = forces["Fx"][start_idx:stop_idx]

                time = time[start_idx:-1]
                Fx = forces["Fx"][start_idx:-1]

                f_input = (2 * np.pi) / config.wavePeriod
                #make regular timestamp and filter force signal
                time_regular = np.linspace(time[0], time[-1], len(time))
                Fx_regular = np.interp(time_regular, time, Fx)
                Fx_filter2 = pymarin_timedom.filtering.fft_filter(Fx_regular, lowpass=2 * f_input, integrate=None,differentiate=None,dt=time_regular[1] - time_regular[0], fs=None,use_extension=True, axis=- 1, minspan=2)



                zero_crossings = np.where(np.diff(np.sign(Fx_filter2)))[0]
                upcrossing = []
                for i in zero_crossings:
                    try:
                        if Fx_filter2[i - 3] < 0 and Fx_filter2[i + 3] > 0:  # check if it is upcrossing
                            upcrossing.append(time_regular[i])
                    except:
                        continue

                try:
                    fully_dev_upcrossing_idx = np.abs(np.asarray(upcrossing) - 75).argmin()
                    start = upcrossing[fully_dev_upcrossing_idx]
                    end = upcrossing[-1]
                    start_idx = np.abs(time - start).argmin()
                    stop_idx = np.abs(time - end).argmin()

                    # now start and end values are known, restart the whole process
                    time = time[start_idx:stop_idx]
                    Fx = Fx[start_idx:stop_idx]
                    time_regular = np.linspace(time[0], time[-1], len(time))
                    Fx_regular = np.interp(time_regular, time, Fx)
                    Fx_filter2 = pymarin_timedom.filtering.fft_filter(Fx_regular, lowpass=2 * f_input, integrate=None,differentiate=None,dt=time_regular[1] - time_regular[0], fs=None,use_extension=True, axis=- 1, minspan=2)
                    #Fx_filter_withouth_low = pymarin_timedom.filtering.fft_filter(Fx_regular, highpass=0.1 * f_input, integrate=None,differentiate=None,dt=time_regular[1] - time_regular[0], fs=None,use_extension=True, axis=- 1, minspan=2)

                    #Fx_mean_withouth_low = np.mean(Fx_filter_withouth_low)
                    Fx_mean = np.mean(Fx_regular)
                    config.mean_wave_drift_force = Fx_mean

                except:
                    Fx_mean = np.mean(Fx)
                    config.mean_wave_drift_force = Fx_mean
                    print('configuration {} had no zero-crossings, so mean off raw force signal taken'.format(config.number))

                if show_all_forcefigures==1:
                    plt.figure()
                    plt.title('configuration_{}'.format(config.number))
                    plt.plot(time, Fx_regular, label='unfiltered force signal', color='r')
                    plt.plot(time_regular, Fx_filter2, label='filtered force signal')
                    #plt.plot(time_regular, Fx_filter_withouth_low, label='filtered low freq force signal', color='green')
                    plt.axhline(y=Fx_mean, color='r', linestyle='-', label='mean')
                    #plt.axhline(y=Fx_mean_withouth_low, color='green', linestyle='-', label='mean_without low')
                    plt.legend()
                    plt.show()

        if show_figures==1:
            plt.figure(figsize=(10,8))
            # manager = plt.get_current_fig_manager()
            # manager.full_screen_toggle()
            plt.title('Forces of {}'.format(run))
            plt.grid()
            for x, singleForce in enumerate(resultDict[indexRun].configuration):
                if singleForce.mean_wave_drift_force:
                    plt.scatter(x, singleForce.mean_wave_drift_force)
                    plt.text(x, singleForce.mean_wave_drift_force, singleForce.number)
            plt.xlabel('configuration')
            plt.ylabel('force [N]')
            plt.show()

    return resultDict

def transmitted_wave_height(resultDict, runs, cutoffSeconds, show_figures):

    for indexRun, run in enumerate(runs):

        wave_onlyFolder = glob.glob(resultDict[indexRun].path + '/wave_only')[0]
        already_calculated=0

        for config in resultDict[indexRun].configuration:
            # IF STABLE, THEN CALCULATE TRANSMITTED WAVE HEIGHT:
            if config.stable:
                filePath = [config.runpath, wave_onlyFolder]

                start_time = cutoffSeconds[indexRun]  # [s] time before is thrown away

                #bounds where transmitted wave height is averaged from
                xmin_transmission = 450
                xmax_transmission = 580


                pathList =[]
                WH=[]
                t=[]
                stop=[]
                #extract data from .h5 file
                for path in filePath:
                    hf = h5py.File(path +'/proc_1/waterheight.h5')
                    WH.append(hf["data"]["wh"][:])
                    t.append(hf["data"]["t"][:])
                    x = hf["data"]["x"][:]
                    pathList.append(path)
                    stop.append((np.abs(t[-1] - start_time)).argmin())

                i_1 = np.argmax(x >= xmin_transmission)
                i_2 = np.argmax(x >= xmax_transmission)


                #LATER MOVE TO SEPERATE FIGURE-FUNCTION!
                if show_figures==1:
                    plt.figure()
                    for i, file in enumerate(filePath):
                        plt.plot(t[i][stop[i]:-1], WH[i][:, 0, i_1][stop[i]:-1], label="{}".format(file))
                    plt.title('water elevation on x = {} m'.format(int(xmin_transmission)))
                    plt.xlabel('Time [s]')
                    plt.ylabel('H [m]')
                    plt.legend()
                    plt.show()

                    plt.figure()
                    for i, file in enumerate(filePath):
                        plt.plot(t[i][stop[i]:-1], WH[i][:, 0, i_2][stop[i]:-1], label="{}".format(file))
                    plt.title('water elevation on x = {} m'.format(int(xmax_transmission)))
                    plt.xlabel('Time [s]')
                    plt.ylabel('H [m]')
                    plt.legend()
                    plt.show()


                segment = [start_time, config.tmax]
                T_p = config.wavePeriod
                if not ((segment[1] - segment[0]) / T_p).is_integer():
                    segment[1] = int(np.floor((segment[1] - segment[0]) / T_p) * T_p + segment[0])

                for filecount, wh in enumerate(WH):
                    if 'wave_only' in filePath[filecount] and already_calculated==0:
                        time = t[filecount]
                        H=[]
                        for j in range(i_1, i_2):  # loop over all x positions


                            eta = wh[:, 0, j]

                            dt = 0.01
                            time_i = np.arange(segment[0], segment[1],dt)  # time vector for interpolation (NB should be within the boundaries set by vector t)
                            f = interpolate.interp1d(time, eta)
                            eta_i = f(time_i)  # interpolated water surface

                            # ---- Compute ensemble-average:
                            etamat = eta_i.reshape(round((segment[1] - segment[0]) / T_p), round(T_p / dt))
                            t_ens = np.arange(0, T_p, dt)  # t of ensemble-averaged matrix
                            eta_ens = np.mean(etamat, axis=0)  # Ensemble-averaged matrix
                            eta_ens_std = np.std(etamat, axis=0)  # Standard deviation over ensemble-averaged matrix

                            # Calculate statistics
                            H_ens = np.max(eta_ens) - np.min(eta_ens)  # Wave height
                            eta_rms = np.std(eta_ens)

                            H.append(H_ens) # all wave heights of every position between xmin_transmission and xmax_transmission
                        H_i =  np.mean(H)
                        already_calculated = 1

                        #config.H_i = H_i


                    if 'configuration' in filePath[filecount]:
                        time = t[filecount]
                        H = []
                        for j in range(i_1, i_2):  # loop over all x positions

                            eta = wh[:, 0, j]

                            dt = 0.01
                            time_i = np.arange(segment[0], segment[1],
                                               dt)  # time vector for interpolation (NB should be within the boundaries set by vector t)
                            f = interpolate.interp1d(time, eta)
                            eta_i = f(time_i)  # interpolated water surface

                            # ---- Compute ensemble-average:
                            etamat = eta_i.reshape(round((segment[1] - segment[0]) / T_p), round(T_p / dt))
                            t_ens = np.arange(0, T_p, dt)  # t of ensemble-averaged matrix
                            eta_ens = np.mean(etamat, axis=0)  # Ensemble-averaged matrix
                            eta_ens_std = np.std(etamat, axis=0)  # Standard deviation over ensemble-averaged matrix

                            # Calculate statistics
                            H_ens = np.max(eta_ens) - np.min(eta_ens)  # Wave height
                            eta_rms = np.std(eta_ens)

                            H.append(H_ens)  # all wave heights of every position between xmin_transmission and xmax_transmission
                        H_t = np.mean(H)

                        config.H_t = H_t

                config.H_i=H_i
                K_t = H_t/H_i
                print('configuration {} has K_t = {}'.format(config.number, K_t))

    return resultDict



def extract_result_data(run):
    with open('../output_files/data_{}'.format(run), 'rb') as f:
        data = pickle.load(f)

    return data


def mean_wave_drift_force_floating_island_strip(H_incident):

    W = 45  # [m] width island modules
    h = 23  # [m] water depth
    d = 8  # [m] draught island modules
    T = 10.4  # [s] wave period
    L = define_L(T)  # [m] wave length
    k = (2 * np.pi) / L  # [1/m] wave







    H_i = []  # list of wave heights in front of each individual module
    H_t = []
    H_r = []
    F = []  # list of forces on each island module

    H_i.append(H_incident)

    # Macagno formula
    K_t_mg = 1. / np.sqrt(1 + ((k * W * np.sinh(k * h)) / (2 * np.cosh(k * h - k * d))) ** 2)

    for i in [1, 2, 3, 4, 5, 6, 7, 8, 9]:
        H_i.append(float(K_t_mg * H_i[i - 1]))
        H_t.append(H_i[-1])
        H_r.append(np.sqrt(H_i[i - 1] ** 2 - H_t[-1] ** 2))

        #print('Island module {} has H_i = {}, H_r = {} and H_t = {}'.format(i, round(H_i[i - 1], 2), round(H_r[i - 1], 2),round(H_t[i - 1], 2)))

        F.append(float((1 / 16) * (H_i[i - 1] ** 2 + H_r[i - 1] ** 2 - H_t[i - 1] ** 2) * 1025 * 9.81 * (1 + (2 * k * h) / np.sinh(2 * k * h))))
        #print('Island module {} experiences {} N of mean wave drift force'.format(i, round(F[-1], 2)))

    total_F = sum(F) * 0.8  # per unit width

    return total_F


\end{lstlisting}
\end{mdframed}









\lstset{language=python,
                basicstyle=\ttfamily,
                numbers=left,
                breaklines=true,
                stepnumber=1,
                showstringspaces=false,
                tabsize=1,
                breakatwhitespace=false,
                keywordstyle=\color{blue}\ttfamily,
                stringstyle=\color{red}\ttfamily,
                commentstyle=\color{applegreen}\ttfamily,
}
\begin{mdframed}[backgroundcolor=light-gray, roundcorner=10pt,leftmargin=1, rightmargin=1, innerleftmargin=0, innertopmargin=7,innerbottommargin=0, outerlinewidth=1, linecolor=light-gray]
\begin{lstlisting}[linewidth=\columnwidth,caption=Functions used in setting up simulations ., label=script:setting up simulations]

    
def geometry_costs_function(T,W,front_fraction,top_fraction,verschuiving,WL,mass_bw,h,x):

    under_slope_area = 0
    right_area = 0
    dx = x[1]-x[0]
    for i, single_h in enumerate(h):
        under_slope_area += abs(h[i])*dx

    block_area_bw = W*T

    right_area+= W*top_fraction* (T - T*front_fraction)
    under_area_bw = W*T*front_fraction

    total_area = (under_area_bw+right_area+under_slope_area)

    mass_steel = 5407/9300 * mass_bw        #source:s@s
    mass_concrete = 3893/9300 * mass_bw     #source:s@s

    fraction_area_2steel =      (right_area+under_area_bw)  / (total_area)         #fraction of area where steel is 2 euro/kg
    fraction_area_var_steel =   (under_slope_area)          / (total_area)          #fraction area where steel is varying euro/kg

    #calculation height variable costs (from 2 to 4 eur/kg, source: S@S)
    alpha = np.arctan((T-T*front_fraction)/(W-W*top_fraction)) * (180/np.pi)
    if alpha<45:
        var_steel_price = 4-(2/45)*alpha
    elif alpha>=45:
        var_steel_price = (2/45) * alpha

    costs_steel = mass_steel*(2*fraction_area_2steel + var_steel_price*fraction_area_var_steel)
    costs_concrete = mass_concrete * 0.775

    geometry_costs = costs_steel + costs_concrete

    return geometry_costs



def define_cg(L,h):

    k = 2*np.pi/L
    cp = np.sqrt((9.81/k)*np.tanh(k*h))
    cg=0.5*cp*(1+ (2*k*h)/np.sinh(2*k*h))

    return cg, k

def define_L(t): #determine the wave length to normalise x in plot, input=period
    def f(L, T=t):
        return (9.81*T**2)/(2*np.pi) * np.tanh(2*np.pi*23/L) - L
    L = fsolve(f,  70)
    return L
    
\end{lstlisting}
\end{mdframed}

