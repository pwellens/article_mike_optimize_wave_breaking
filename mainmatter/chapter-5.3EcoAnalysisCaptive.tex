\section{Discussion on Economic Optimisation}


The main objective from this thesis is to reduce the costs of a floating island by coupling a breakwater to its wave-ward side, which attenuated wave energy, while minimising its own mean wave drift force, and thereby reducing the costs on the required mooring system. To provide this profitable structure, the construction costs of the breakwater must be less than its reduction in the mooring costs it produces. Therefore, a cost function (explained in Section \ref{sec: cost analysis methodology }) has been written that translates the hydrodynamic performances of the breakwaters: mean wave drift force and wave transmission, into costs. Therefore, every breakwater in the design space resulted in a certain total reduction in costs, which was used to perform an economic design optimisation.\\
\\
This resulted in different optima as found by the hydrodynamic optimisation. Breakwaters that performed best at attenuating waves while having a low mean wave drift force were simply too large and, therefore, too expensive. The optimum based on costs converged to the lowest width of 10 metres and has a relatively large depth of 9.6 metres. The breakwater crosses the water surface and has a draught of 6.16 metres. According to the ComFLOW output, together with the cost function presented, this breakwater will lead to a cost reduction of \texteuro 24.305 per unit width. Therefore, for the total structure (of nine sections wide, each with a width of 45 m), this leads to a cost reduction of 9.8 M\texteuro. Its position with respect to the waterline is consistent with the observation of the hydrodynamic design optimisation that for wedge-type breakwaters with a small width $W$, the optimal position of the top of the breakwater (to minimise the transmission coefficient) is metres above the water surface (see response surface \ref{fig: Fd_W_WL_wedge DI1 H3 captive}). \\
\\
The cost function presented in this research is simplistic, so it will not provide an exact amount on the profit of the optimal breakwater. But it shows that the construction costs of a breakwater made of steel and concrete are of the same order of magnitude as the reduction in mooring costs due to its presence. According to this cost function, some breakwaters in the design space appear to provide a cost reduction, indicating that the relevance of the research is high. Some breakwaters appear to be loss-making, indicating that further research is required to a more accurate cost function when planning to design an actual breakwater connected to a floating island. \\
\\
There are several arguments for why the cost functions are conservative and others for why it is progressive. The cost function is progressive because:
\begin{itemize}
    \item Materials prices change over time.
    \item In reality, the breakwater has to be connected to the floating island with fenders. The costs of these fenders are not included in the cost analysis.
    \item Maintenance expenses on the breakwaters and fenders are not included.
    \item The noise of breaking waves may decrease its workability. 
\end{itemize}
And the same cost function is conservative because:
\begin{itemize}
    \item A lower transmitted wave height will generally also decrease the motions of the floating island, increasing its workability, which is quite valuable intuitively. But it is hard to quantify because it depends on i.a. the dimensions of the island and its purpose, and may differ from module to module. 
    \item The extra structure of the breakwater may be able to serve other purposes as well: energy generation, growing food (e.g. oysters), living, working, storage, and water slide.
\end{itemize}

It is good to note that the optimum found is based on simulations in which motions on the structure are restricted, a so-called 'captive' setup. This probably has a great effect on the optimum found. Structures with a large draught and a large area that crosses the waterline have good pitch-restoring capacities. In other words, these kinds of structures will naturally be more stable. This influences the drift force and wave transmission and might have a great impact on the geometry of the optimal breakwater found.